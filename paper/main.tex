\documentclass{article}

\usepackage[margin=1in]{geometry}
\usepackage{amsmath, amssymb, amsthm}
\usepackage{mathtools}


\usepackage[round]{natbib}   % omit 'round' option if you prefer square brackets
\bibliographystyle{plainnat}

\usepackage{hyperref}
\hypersetup{
  colorlinks   = true, %Colours links instead of ugly boxes
  urlcolor     = blue, %Colour for external hyperlinks
  linkcolor    = blue, %Colour of internal links
  citecolor    = blue  %Colour of citations
}

\newtheorem{claim}{Claim}
\newtheorem{corollary}{Corollary}
\newtheorem{definition}{Definition}
\newtheorem{example}{Example}
\newtheorem{lemma}{Lemma}
\newtheorem{question}{Question}
\newtheorem{theorem}{Theorem}

\DeclareMathOperator*{\argmin}{arg\,min}
\DeclareMathOperator*{\argmax}{arg\,max}
\DeclareMathOperator*{\vecspan}{span}
\DeclareMathOperator*{\affspan}{aff}
\DeclareMathOperator*{\subG}{subG}
\DeclareMathOperator*{\tr}{tr}
\DeclareMathOperator*{\E}{\mathbb{E}}

\newcommand{\str}[1]{\texttt{#1}}
\newcommand{\defn}[1]{\textit{#1}}

\newcommand{\N}{\mathbb{N}}
\newcommand{\R}{\mathbb{R}}
\newcommand{\trans}[1]{{#1}^{\top}}

\newcommand{\abs}[1]{\lvert{#1}\rvert}
\newcommand{\ltwo}[1]{\lVert {#1} \rVert_2}

\newcommand{\w}{\mathbf w}
\newcommand{\what}{\hat\w}
\newcommand{\x}{\mathbf{x}}
\newcommand{\y}{y}

%\newcommand{\relu}{\sigma_{\text{RELU}}}
\newcommand{\relu}{\sigma}

\newcommand{\ignore}[1]{}
\newcommand{\fixme}[1]{\textcolor{red}{\textbf{FIXME:} {#1}}}

\begin{document}

\section{Introduction}\label{sec:intro}
In this article, we'll describe how the ResNet's matrix operations affect data and holes therein.
We'll define basic topological and metric concepts in Section \ref{sec:metric}.
Then, we'll define our matrx operations and describe how they affect holes in data in Section \ref{sec:holes}.
We'll use this in Section \ref{sec:result} to describe what role they play in neural net classification.

\subsection{Metric Spaces}\label{sec:metric}
The category \texttt{Met} of metric spaces has metric spaces as its objects and metric homomorphisms (continuous functions that do not increase pairwise distance) as its morphisms.

For some $n\in\N$, let $f:(\R^n,d)\to(\R^n,d)$ be an automorphism between metric spaces with metric outer measure $\mu$.

\begin{definition}
For a given positive integer $0\le k<n$, define the \emph{open $k$-ball} of radius $\epsilon\in\R_{>0}$ centered at $q\in\R^{k+1}$ to be $B_d(q,\epsilon)=\{p\in\R^{k+1} \mid d(q,p)< \epsilon \}$.
% The $k$-sphere $S_k$ is its boundary, and the closed $k$-ball is their union.
\end{definition}

\begin{lemma}\label{lem:bigger}
Let $Y\subset \R^n$ be the image of a continuous injective map from $B_d(q,\epsilon_0)$. Then, $\mu(f(X))\le \mu(Y)$.
\end{lemma}
\begin{proof}
Suppose $f$ makes $Y$ bigger:
$$\mu(Y)<\mu(f(Y)).$$
Then, there's an open ball in $Y$ which gets bigger:
\begin{align}\label{eq:bigger}
    \exists q\in Y, \delta\in \R_{>0} \text{ such that } \mu(B_d(q,\delta))< \mu(f(B_d(q,\delta)))
\end{align}
Since $f$ is a morphism, it decreases the distance between the center and any other point in the ball:
$$\forall p\in B_d(q,\delta),  d(f(q), f(p))\le d(q,p).$$
Thus, the image of ball lies inside a ball of the same size:
\begin{align*}
    f(B_d(q,\delta)) &\subset B_d(f(q), \delta), \text{ which implies }\\
    \mu(f(B_d(q,\delta)) &\le \mu(B_d(f(q), \delta)) \text{ since $\mu$ is a measure.}
\end{align*}
Since this contradicts Inequality \ref{eq:bigger}, $f$ can only make $Y$ smaller.
\begin{align*}
    \mu(Y) &\nless \mu(f(Y)) \text{, or equivalently} \\
    \mu(f(Y)) &\le \mu(Y).
\end{align*}
\end{proof}
Intuitively, this lemma says that morphisms in the category of metric spaces with metric outer measures maintain or decrease the measure of a given compact subspace.

\begin{definition}
A \emph{$k$-dimensional hole} is the bounded $(k+1)$-dimensional component of $\R^n-\partial Y$, where $\partial Y\subset \R^{k}$ is the boundary of the image of a continuous injective map from $ B_d(q,\epsilon_0)$.
\end{definition}

In the following sections, we'll prove that the rectifier, among other activation functions, is a morphism and use our lemma to explain what it does to $k$-dimensional holes.

\section{Homological Side Effects}\label{sec:holes}
We'll discuss activation functions Section \ref{sec:act} and max pool in Section \ref{sec:pool}.

\subsection{Activation Functions}\label{sec:act}
\subsubsection{ReLU}
\begin{definition}
The \emph{ReLU activation function} $\sigma:(\R^n,||\cdot||_2)\to(\R^n,||\cdot||_2)$ is defined as the positive parts of its arguments, where $\R^n$ has the usual ordering.
\[
(y_1,\dots,y_n)\mapsto (\max(0, y_1), \dots, \max(0,y_n))
\]
\end{definition}

\begin{lemma}\label{lem:relu}
ReLU is a morphism in the category of metric spaces.
\end{lemma}

\begin{proof}
Let $x_1,x_2\in (\R^n,||\cdot||_2)$, where $\R^n$ has the usual ordering.
First, we'll prove what ReLU does to the $j$th component in three cases, where each case is a possible pair of signs.

\begin{enumerate}
    \item If $x_{1_j},x_{2_j}\ge0$, then $\sigma(x_{1_j})=x_{1_j}$ and $\sigma(x_{2_j}) = x_{2_j}$. Thus, $(\sigma(x_{1_j})-\sigma(x_{2_j}))^2 = (x_{1_j}-x_{2_j})^2$.
    \item If $x_{1_j},x_{2_j}\le0$, then $\sigma(x_{1_j})=0$ and $\sigma(x_{2_j})=0$. Thus, $(\sigma(x_{1_j})-\sigma(x_{2_j}))^2=(0-0)^2=0\le (x_{1_j}-x_{2_j})^2$.
    \item Assume $x_{1_j}\ge 0, x_{2_j}\le0$ without loss of generality. 
     Since $(x_{1_j}-0)\le (x_{1_j}-x_{2_j})$, $(\sigma(x_{1_j})-\sigma(x_{2_j}))^2=(x_{1_j}-0)^2 \le (x_{1_j}-x_{2_j})^2$.
\end{enumerate}
In each of three cases,
\[
(\sigma(x_{i_k})-\sigma(x_{j_k}))^2 \le (x_{1_j}-x_{2_j})^2.
\]
Since each term is dominated,
\begin{align*}
    \sum_{k=1}^d (\sigma(x_{i_k})-\sigma(x_{j_k}))^2 &\le \sum_{k=1}^d (x_{i_k}-x_{j_k})^2,\\
     \sqrt{\sum_{k=1}^d (\sigma(x_{i_k})-\sigma(x_{j_k}))^2} &\le \sqrt{\sum_{k=1}^d (x_{i_k}-x_{j_k})^2},\\
     \ltwo{\sigma(\x_i)-\sigma(\x_j)} &\le \ltwo{\x_i-\x_j}.
\end{align*}.
Lastly, $\relu$ is continuous since it's a component-wise application of the continuous function $\max$.
\end{proof}

According to Lemma \ref{lem:bigger}, ReLU transforms any set of vectors $X=\{\x_1,...,\x_m\}\subset \R^n$, which define a $k$-dimensional hole, into a hole equal or smaller in size.

\begin{question}
How does ReLU affect metric entropy?
\end{question}
% https://izbicki.me/public/papers/dissertation.pdf

\subsubsection{Others}
Any other morphism in the category of metric spaces is non-increasing on $k$-dimensional holes.
The authors believe that any activation function which preserves order and maps to a smaller object is a morphism.

\subsubsection{Matrix Representation}
Alternately, you can view activation functions as mapping $m\times n$ matrices to $m\times n$ matrices, component-wise.

\subsection{Max Pool}\label{sec:pool}
Let $X=\{x_1,\dots,x_m\}\subset(\R^n,d)$ be represented by an $m\times n$ matrix for some even $m,n\in 2\N$.
Consider a $3\times 3$ pool $p$ with stride $2$.
Then, $p:\R^{mn}\to\R^{\frac{m}{2}\frac{n}{2}}$ maps $m\times n$ matrices to $\frac{m}{2}\times\frac{n}{2}$ matrices.
\begin{definition}
max pool
\end{definition}
You can think about max pooling over a row, and then max pooling over a column. Is that equivalent and does it help?
\begin{question}

\end{question}


\section{ResNet}\label{sec:result}
ResNet composes many operations, including convolutions, ReLUs, and pools, which may affect holes in different ways, including, creating, destroying, growing, shrinking, and identity. 

\begin{question}
How does max pool affect pairwise distances in our data?
\end{question}

\begin{question}
How does this apply to neural networks like ResNet?
\end{question}
\begin{enumerate}
\item We have described what ReLU does to data, but we have yet to describe what the other operations do.
\item It remains unclear what roles these play in learning.
\end{enumerate}

\bibliography{main}


\end{document}

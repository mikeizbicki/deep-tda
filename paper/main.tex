\documentclass{article}

%%%%%%%%%%%%%%%%%%%%%%%%%%%%%%%%%%%%%%%%%%%%%%%%%%%%%%%%%%%%%%%%%%%%%%%%%%%%%%%%

\usepackage[margin=1in]{geometry}
\usepackage{amsmath}
\usepackage{amssymb}
\usepackage{amsthm}

\usepackage[round]{natbib}   % omit 'round' option if you prefer square brackets
\bibliographystyle{plainnat}

\usepackage{hyperref}
\hypersetup{
  colorlinks   = true, %Colours links instead of ugly boxes
  urlcolor     = blue, %Colour for external hyperlinks
  linkcolor    = blue, %Colour of internal links
  citecolor    = blue  %Colour of citations
}

%%%%%%%%%%%%%%%%%%%%%%%%%%%%%%%%%%%%%%%%

\newtheorem{corollary}{Corollary}
\newtheorem{claim}{Claim}
\newtheorem{theorem}{Theorem}
\newtheorem{example}{Example}
\newtheorem{lemma}{Lemma}
\newtheorem{question}{Question}
\newtheorem{fact}{Fact}

\DeclareMathOperator*{\argmin}{arg\,min}
\DeclareMathOperator*{\argmax}{arg\,max}
\DeclareMathOperator*{\vecspan}{span}
\DeclareMathOperator*{\affspan}{aff}
\DeclareMathOperator*{\subG}{subG}
\DeclareMathOperator*{\tr}{tr}
\DeclareMathOperator*{\E}{\mathbb{E}}

\newcommand{\str}[1]{\texttt{#1}}
\newcommand{\defn}[1]{\textit{#1}}

\newcommand{\N}{\mathbb{N}}
\newcommand{\R}{\mathbb{R}}
\newcommand{\trans}[1]{{#1}^{\top}}

\newcommand{\abs}[1]{\lvert{#1}\rvert}
\newcommand{\ltwo}[1]{\lVert {#1} \rVert}

\newcommand{\w}{\mathbf w}
\newcommand{\what}{\hat\w}
\newcommand{\x}{\mathbf x}
\newcommand{\y}{\mathbf y}

%\newcommand{\relu}{\sigma_{\text{RELU}}}
\newcommand{\relu}{\sigma}

\newcommand{\ignore}[1]{}
\newcommand{\fixme}[1]{\textcolor{red}{\textbf{FIXME:} {#1}}}
\newcommand{\todo}[1]{{\textbf{TODO:} {#1}}}

\newcommand{\emin}{\lambda_{\text{min}}}
\newcommand{\emax}{\lambda_{\text{max}}}

\DeclareMathOperator{\maxpool}{\textup{\texttt{maxpool}}}
\DeclareMathOperator{\ddimop}{\textup{dim}}
\newcommand{\ddim}[1]{\ddimop(#1)}
\newcommand{\covnum}[2]{\mathcal N_{#1}(#2)}
\newcommand{\rips}[2]{\mathcal R_{#1}(#2)}
\newcommand{\pd}[1]{\mathcal {PD}(#1)}
\newcommand{\elem}[2]{{#2}^{(#1)}}


%%%%%%%%%%%%%%%%%%%%%%%%%%%%%%%%%%%%%%%%%%%%%%%%%%%%%%%%%%%%%%%%%%%%%%%%%%%%%%%%

\begin{document}

%%%%%%%%%%%%%%%%%%%%%%%%%%%%%%%%%%%%%%%%%%%%%%%%%%%%%%%%%%%%%%%%%%%%%%%%%%%%%%%%

%%%%%%%%%%%%%%%%%%%%%%%%%%%%%%%%%%%%%%%%%%%%%%%%%%%%%%%%%%%%%%%%%%%%%%%%%%%%%%%%

\section{What We Know}

Let $X=\{\x_1,...,\x_n\}$, where each $\x_i\in\R^d$.
Without a subscript, $\x$ is an arbitrary $\x_i$.

\begin{fact}
    Let $A : \R^{e\times d}$ with minimum and maximum eigenvalues $\emin$ and $\emax$.
    Then,
    \begin{equation}
        \emin\ltwo{\x} \le \ltwo{A\x} \le \emax\ltwo{\x}.
    \end{equation}
\end{fact}

\begin{lemma}
    Let $\sigma:\R\to\R$ be a $\lambda$-Lipschitz activation function.
    Then,
    \begin{equation}
        \ltwo{\sigma(\x_i)-\sigma(\x_j)}
        \le
        \lambda\ltwo{\x_i-\x_j} 
    \end{equation}
    for all $i$, and $j$.
    In particular, $\sigma$ may be the relu, leaky relu, or sigmoid activation functions.
\end{lemma}

\begin{proof}
    We have that
    \begin{align}
        \ltwo{\sigma(\x_i) - \sigma(\x_j)}^2 
        &=
        \sum_{k=1}^d \abs{\sigma(\elem k\x_i) - \sigma(\elem k\x_j)}^2
        \\&\le
        \sum_{k=1}^d (\lambda\abs{\elem k\x_i - \elem k\x_j})^2
        \\&=
        \lambda^2\sum_{k=1}^d \abs{\elem k\x_i - \elem k\x_j}^2
        %\sum_{k=1}^d (\sigma(\elem k\x_i - \elem k\x_j))^2
        %\\&\le
        %\sum_{k=1}^d (\elem k\x_i - \elem k\x_j)^2
        \\&= 
        \lambda^2\ltwo{\x_i - \x_j}^2 
        .
    \end{align}
    Taking the square root of both sides gives the stated result.
\end{proof}

\begin{lemma}
    Let $d$ be a multiple of 2, and $\maxpool : \R^d \to \R^{d/2}$ be the max pooling function.
    That is,
    \begin{equation}
        \elem{i}{\maxpool(\x)} = \max\{\elem{2i}\x,\elem{2i+1}\x\}
        .
    \end{equation}
    If $\elem k\x\ge0$ for all $k$, then
    \begin{equation}
        \tfrac12\ltwo{\x} \le \ltwo{\maxpool(\x)} \le \ltwo{\x}.
    \end{equation}
\end{lemma}

\begin{proof}
    We can write
    \begin{equation}
        \ltwo{\maxpool(\x)}^2
        =
        \sum_{k=1}^{d/2} {\elem k {\maxpool(\x)}}{}^2
        =
        \sum_{k=1}^{d/2} \max\{\elem{2k}\x,\elem{2k+1}\x\}^2
        \label{eq:pf:maxpool}
        .
    \end{equation}
    To get the rightmost inequality, we apply the fact that $\max\{a,b\}^2 \le a^2 + b^2$ for all $a,b\ge0$ to \eqref{eq:pf:maxpool} to get
    \begin{align}
        \ltwo{\maxpool(\x)}^2
        \le
        \sum_{k=1}^{d} (\elem{k}\x)^2
        =
        \ltwo{\x}^2
        .
    \end{align}
    For the left inequality,
    we apply the fact that $\max\{a,b\} \ge \tfrac12(a+b)$ to \eqref{eq:pf:maxpool} to get
    \begin{align}
        \ltwo{\maxpool(\x)}^2
        &\ge
        \sum_{k=1}^{d/2} \tfrac14(\elem{2k}\x+\elem{2k+1}\x)^2
        \\&=
        \sum_{k=1}^{d/2} \tfrac14((\elem{2k}\x)^2+2\elem{2k}\x\elem{2k+1}\x+(\elem{2k+1}\x)^2)
        \\&\ge
        \sum_{k=1}^{d/2} \tfrac14((\elem{2k}\x)^2+(\elem{2k+1}\x)^2)
        \\&=
        \tfrac14\ltwo{\x}^2
        .
    \end{align}
    Taking the square root of both sides gives the stated result.
\end{proof}

%%%%%%%%%%%%%%%%%%%%%%%%%%%%%%%%%%%%%%%%%%%%%%%%%%%%%%%%%%%%%%%%%%%%%%%%%%%%%%%%

\section{Working on}

\begin{question}
    Let $\covnum\epsilon X$ denote the $\epsilon$-covering number of $X$.
    What is the relationship between $\covnum\epsilon X$, $\covnum\epsilon {AX}$, $\covnum\epsilon {\sigma X}$, and $\covnum\epsilon{\maxpool X}$?
\end{question}

\fixme{
\begin{lemma}
    Let $A : \R^{e\times d}$ satisfying XXX. 
    Then,
    \begin{equation}
        \covnum\epsilon X 
        ~~~?~~~ 
        \covnum\epsilon{AX}
        .
    \end{equation}
\end{lemma}
}

%%%%%%%%%%%%%%%%%%%%%%%%%%%%%%%%%%%%%%%%%%%%%%%%%%%%%%%%%%%%%%%%%%%%%%%%%%%%%%%%

\section{Things we don't know}

\begin{question}
    Let $\ddim X$ denote the doubling dimension of $X$.
    What is the relationship between $\ddim X$, $\ddim {AX}$, $\ddim {\sigma X}$, and $\ddim{\maxpool X}$?
\end{question}

\begin{question}
    Let $\rips\epsilon X$ denote the Rips complex of $X$.
    What is the relationship between $\rips\epsilon X$, $\rips\epsilon {AX}$, $\rips\epsilon {\sigma X}$, and $\rips\epsilon{\maxpool X}$?
\end{question}

\begin{question}
    Let $\pd\epsilon X$ denote the persistence diagram of $X$.
    What is the relationship between $\pd X$, $\pd{AX}$, $\pd{\sigma X}$, and $\pd{\maxpool X}$?
\end{question}

%%%%%%%%%%%%%%%%%%%%%%%%%%%%%%%%%%%%%%%%%%%%%%%%%%%%%%%%%%%%%%%%%%%%%%%%%%%%%%%%

\section{Related Work}

\subsection{machine learning + TDA}

\citet{chen2019topological} introduce a differentiable regularization term based on the 0-homology of data.
The regularization term is a piecewise linear approximation to a norm of the persistence diagram of the classifier's boundary.
They perform experiments on kernel logistic regression against other functional and geometric regularizers,
but it's not clear how to extend this to deep neural networks.

\citet{rieck2019neural} present a theory of \defn{neural persistence} (NP),
which uses 0-homology persistence diagrams to characterize the capacity of neural networks.
For each layer in the graph,
they construct a persistence diagram based on the layer's node weights,
and the NP is the $p$-norm of this diagram.
They provide two theorems that bound a network's NP,
but there is no theoretical result on how the neural persistence affects a network's performance.
They introduce a heuristic stopping rule based on the NP and suggest using this as an alternative to a validation set,
but their experiments are pretty limited.

It is well known that shallow but wide neural networks can universally approximate any function over a compact space.
\citet{johnson2019deep} shows that deep but narrow networks are not universal approximators using techniques from TDA to analyze the topological properties of the network.
(Many similar papers have been submitted and rejected; see: 
\url{https://openreview.net/forum?id=r1VPNiA5Fm}, 
\url{https://openreview.net/forum?id=B1TTpYKgx}.)

\citet{guss2018characterizing,ramamurthy2018topological} claim to bound the capacity of deep neural networks,
but haven't been published yet.

\subsection{theory of TDA}

Theorem 4 of \citet{chazal2015convergence} provides a general tool for bounding the expected error of a persistence diagram.

\todo{I need to review the following in more detail to find the specific results that may be useful.}
\citet{chazal2014stochastic} provides tools for persistence landscapes.
\citet{chazal2012structure} is a book on the stability of persistence diagrams.
\citet{chazal2015subsampling} propose what amounts to a divide and conquer approach to computing persistence diagrams.
Their theoretical results likely have more broad applications.
\citet{chazal2018density} use kernel functions to provide bounds on the convergence of persistence diagrams to their expectation.

\subsubsection{embeddings}

Persistence diagrams can be equipped with the \defn{diagram distance} to form a metric space.
Many works have attempted to linearize this metric space \citep[e.g.][]{zielinski2018persistence,le2018persistence,hofer2017deep,kusano2016persistence,carriere2017sliced,cang2017topologynet,anirudh2016riemannian,obayashi2018persistence,adams2017persistence,kwitt2015statistical}.
The goal of most work along these lines is to generate features from persistence diagrams,
and use these features as input to machine learning algorithms.
In contrast, \citet{chazal2018density} use kernel functions to provide bounds on the convergence of persistence diagrams to their expectation.
They have a particularly good review of the existing theory on this convergence.

\citet{carriere2018metric} prove that for any such linearization to be bi-Lipschitz, 
the resulting Hilbert space must be infinite dimensional.
They also show that in a number of existing infinite dimensional linearizations,
the Lipschitz bounds depend on the number of components in the persistence diagram,
which is undesirable.

\subsection{theory of deep learning}

\citet{zhang2016understanding,neyshabur2017exploring} ask the question: 
Why do deep neural networks generalize well?
See also: \url{https://openreview.net/forum?id=BJxOHs0cKm}.
\citet{jakubovitz2018generalization} provides an elementary survey of why this is an important question to ask.

Theorem 20.6 of \citet{shalev2014understanding} shows that the VC dimension of a shallow neural network using the sign activation function grows as $O(E\log E)$, 
where $E$ is the total number of parameters in the network.
\citet{bartlett2017nearly} and \citet{harvey2017nearly} show that the VC dimension of deep relu networks grows as $O(WL\log(W))$,
where $W$ is the number of weights per layer and $L$ the number of layers.
This result is nearly tight because the VC dimension for some networks is lower bounded by $\Omega(WL\log(W/L))$.

\citet{zhou2018compressibility} prevents non-trivial PAC bounds based on model compressibility.
\citet{barron2018approximation} presents generalization bounds in terms of the $L_1$ norm of the parameters.
\citet{neyshabur2018role} argues that overparameterization is important for generalization.
\citet{perez2018deep} argues that there is an implicit regularization that exponentially biases models towards simpler explanations.
\citet{mou2018dropout} provide generalization bounds based on dropout.
\citet{zhang2019all} provides empirical evidence that many layers in a network are irrelevant for the final prediction,
and this justifies the fact that the VC dimension is too course a measure of model complexity.

\subsection{images}

\citet{carlsson2008local} show that the space of $3\times3$ image patches have the 2 dimensional topology of a Klein bottle.
The naive ambient space is 9 dimensional,
but 7 dimensional after mean centering and whitening.

\bibliography{main}


\end{document}
